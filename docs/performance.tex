\documentclass[10pt,a4paper]{article}
\usepackage[latin1]{inputenc}
\usepackage{amsmath}
\usepackage{amsfonts}
\usepackage{amssymb}
\usepackage{graphicx}

\begin{document}
\section*{XRT Performance Calculation}

\section{Introduction}

\section{X-ray Source}

"The Crab nebula is the brightest astrophysical source in the $\gamma$-ray sky ($E > 30$ keV)" (add reference). The shape of the spectrum can be found seen in Figure \ref{fig:crabSpectrum}.

\begin{figure}
\centering
\includegraphics[width=0.7\linewidth]{./crabSpectrum}
\caption[$\gamma$-ray spectrum of Crab nebula]{Taken from}
\label{fig:crabSpectrum}
\end{figure}

We can approximate the spectrum as: \[ \frac{\partial N}{\partial E} = N_{0} \left( \frac{E}{1 GeV} \right) ^{-\Gamma} \exp \left( - \frac{E}{E_{cutoff}} \right) \]

Which is valid in the energy range 0.75-300GeV. Best fit values for $N_{0}$ is $4.3 \times 10^{-10} ph cm^{-2} s^{-1} MeV^{-1}$, $\Gamma = 2.20$ and $E_{cutoff} = 77 MeV$. (Insert ref.). 

\section{Atmospheric Absorption}

\section{Detector Performance}
\subsection{Geiger-Mueller Tube}
\subsection{Scintillation Crystal}

\section{System Performance}

\section{Conclusions}

\section{References}

\end{document}
